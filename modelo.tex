\NeedsTeXFormat{LaTeX2e}
%-----------------------------------------------------------
\documentclass[a4paper,12pt]{monografia}
\usepackage[portuguese, colorinlistoftodos, textsize=tiny]{todonotes}
\usepackage{amsmath,amsthm,amsfonts,amssymb}
\usepackage[mathcal]{eucal}
\usepackage{latexsym}
\usepackage[portuguese]{babel}  
\usepackage[utf8]{inputenc}
\usepackage{setspace}
%\usepackage{natbib}
\usepackage{bm}
\usepackage[portuguese,algoruled,longend, linesnumbered]{algorithm2e}
\usepackage{listings}
\usepackage{graphicx}
\usepackage{hyperref}
\hypersetup{colorlinks,
   debug=false,
   linkcolor=black,  %%% cor do tableofcontents, \ref, \footnote, etc
   citecolor=black,  %%% cor do \cite
   urlcolor=black,   %%% cor do \url e \href
   bookmarksopen=true,
}
\usepackage[alf,bibjustif]{abntex2cite}

\newcounter{todocounter}
\newcommand{\comment}[2][]
{\stepcounter{todocounter}\todo[caption={\thetodocounter: #2}, #1] 
{\begin{spacing}{1}\thetodocounter: #2\end{spacing}}}
\reversemarginpar
\setlength{\marginparwidth}{2.5cm}
\lstloadlanguages{C}
%-----------------------------------------------------------
%-----------------------------------------------------------
\theoremstyle{plain}
\newtheorem{theorem}{Teorema}[section]
\newtheorem{axiom}{Axioma}[section]
\newtheorem{corollary}{Corol\'ario}[section]
\newtheorem{lemma}{Lema}[section]
\newtheorem{proposition}{Proposi\c{c}\~ao}[section]
%-----------------------------------------------------------
\theoremstyle{definition}
\newtheorem{definition}{Defini\c{c}\~ao}[section]
\newtheorem{example}{Exemplo}[section]
%-----------------------------------------------------------
\theoremstyle{remark}
\newtheorem{remark}{Observa\c{c}\~ao}[section]
%-----------------------------------------------------------
%-----------------------------------------------------------
\newcommand{\R}{\mathbb{R}}
\newcommand{\N}{\mathbb{N}}
\newcommand{\Z}{\mathbb{Z}}
\newcommand{\Q}{\mathbb{Q}}
\newcommand{\K}{\mathbb{K}}
\newcommand{\I}{\mathbb{I}}
\newcommand{\id}{\mathbf{1}}
\newcommand{\U}{\mathcal{U}}
\newcommand{\V}{{\cal V}}
%-----------------------------------------------------------
\def\ind{\hbox{ ind }}
%-----------------------------------------------------------
\hyphenation{Con-si-de-ra-mos}
\hyphenation{me-lhor}
\hyphenation{res-pos-ta}
\hyphenation{re-qui-si-tos}




\begin{document}
%%%%%%%%%%%%%%%%%%%%%%%%%%%%%%%%%%%%%%%%%%%%%%%%%%%%%%%%%%%5
%
%                  INFORMAÇÕES PRÉ-TEXTUAIS
%
%----------------- Título e Dados do Autor -----------------
\titulo{$<<$T\'itulo da monografia$>>$}
\subtitulo{$<<$Subt\'itulo - opcional$>>$} % opcional
\autor{$<<$Nome do aluno$>>$} \nome{$<<$Primeiro Nome$>>$} \ultimonome{$<<$\'Ultimo Nome$>>$}
%
%---------- Informe o Curso e Grau -----
\mestrado %Pode ser \bacharelado \licenciatura \especializacao \mestrado ou \doutorado
\curso{$<<$Nome do curso$>>$} 
\dia{$<<$dia$>>$} \mes{$<<$m\^es$>>$} \ano{$<<$ano$>>$} % data da aprovação
\cidade{$<<$Juiz de Fora$>>$}
%
%----------Informações sobre a Instituição -----------------
\instituicao{$<<$Nome da universidade$>>$} \sigla{$<<$UFJF$>>$}
\unidadeacademica{$<<$Nome do instituto$>>$}
\departamento{$<<$Nome do departamento$>>$}
%
%------Nomes do Orientador, 1o. Examinador e 2o. Examinador-
\orientador{$<<$Nome do Orientador$>>$}
\ttorientador{$<<$T\'itulo do Orientador$>>$}
%
\coorientador{$<<$Nome do Co-orientador$>>$} % opcional
\ttcoorientador{$<<$T\'itulo do Co-orientador$>>$} % se digitado \coorientador
%
\examinadorum{$<<$Nome do Examinador 1$>>$}
\ttexaminadorum{$<<$T\'itulo do Examinador 1$>>$}
%
\examinadordois{$<<$Nome do Examinador 2$>>$}
\ttexaminadordois{$<<$T\'itulo do Examinador 2$>>$}
%
\examinadortres{Nome do Examinador 3}
\ttexaminadortres{T\'itulo do Examinador 3}
%
\examinadorquatro{Nome do Examinador 4}
\ttexaminadorquatro{T\'itulo do Examinador 4}
%
%-------- Informações obtidas na Biblioteca ----------------
%
%\CDU{536.21} \areas{1.Análise Matemática  2. Topologia.}
%\npaginas{xx}  % total de páginas do trabalho
%
%%%%%%%%%%%%%%%%%%%%%%%%%%%%%%%%%%%%%%%%%%%%%%%%%%%%%%%%
%    DADOS DE FORMATACAO DA MONOGRAFIA
%
% A instrução abaixo insere a logo do curso e da instituicao na capa da monografia. Basta comentar caso não queira os logos. Para alterar o logo da instituicao e curso, basta alterar os arquivos logoInstituicao.png e logoCurso.jpg. Caso deseje alterar os arquivos, os substituta por imagens do mesmo tamanho!
%\inserirlogo  
%
%
%
%          FIM DAS INFORMAÇÕES PRÉ-TEXTUAIS
%
%%%%%%%%%%%%%%%%%%%%%%%%%%%%%%%%%%%%%%%%%%%%%%%%%%%%%%%%%

\maketitle









%----------------------------dedicat\'oria  opcional--------------
\begin{dedicatoria}
Aos meus amigos e irm\~aos.\\
Aos pais, pelo apoio e sustento.\\
\end{dedicatoria}


%--------Digite aqui o seu resumo em Portugu\^es--------------
\resumo{Resumo} As instru\c{c}\~oes aqui contidas objetivam auxiliar os autores na prepara\c{c}\~ao de documentos para impress\~ao de monografias do Departamento de Ci\^encia da Computa\c{c}\~ao. Os estilos encontram-se definidos em um modelo denominado Monografia.cls. O resumo deve ser escrito na mesma l\'ingua do texto (Portugu\^es, Ingl\^es ou Espanhol) e descreve o conte\'udo do texto em cerca de 150-200 palavras. Esta \'e a primeira vers\~ao das instru\c{c}\~oes e dos formatos e, portanto, sujeita a incorre\c{c}\~oes e omiss\~oes. Sugest\~oes de melhorias s\~ao muito benvindas: envie mensagem para jairo.souza@ufjf.edu.br.

\noindent \\ \textbf{Palavras-chave:} Monografia, latex, instru\c{c}\~oes.



%-----------Digite aqui o seu resumo em Ingl\^es--------------
\resumo{Abstract} You must summarize your work in 150-200 words.

\noindent \\ \textbf{Keywords:} Monograph, latex, instructions.


%-----------Ou digite aqui o seu resumo em Frances----------
%\resumo{Resumo} C'est un mod\'ule de la monographie dans \LaTeX et
%5utilise la classe monografia.cls, avec le but de aider dans le
%maniement des travaux de conclusion des plusieurs cours de
%l'Universit\'e F\'ed\'erale de Juiz de Fora.



%-----------------------------------------------------------
\agradecimento{Agradecimentos} \indent\indent 
A todos os meus parentes, pelo encorajamento e
apoio.

Ao professor Beltrano pela orienta\c{c}\~ao, amizade e
principalmente, pela paci\^encia, sem a qual este trabalho n\~ao se
realizaria.


Aos professores do Departamento de Ci\^encia da Computa\c{c}\~ao pelos seus
ensinamentos e aos funcion\'arios do curso, que durante esses anos,
contribu\'iram de algum modo para o nosso enriquecimento pessoal e
profissional.
\newpage


%---------------------- EPÍGRAFE I (OPCIONAL)--------------
\begin{epigrafe}
``Lembra que o sono \'e sagrado e alimenta de horizontes o tempo acordado de viver''.\\
\hfill Beto Guedes (Amor de Índio)
\end{epigrafe}



%----Sum\'ario, lista de figura e de tabela ------------
 \tableofcontents \thispagestyle{empty} \listoffigures
\thispagestyle{empty} \listoftables \thispagestyle{empty}



%----Gloss\'ario ------------
\chapter*{Lista de Abrevia\c{c}\~oes} \addcontentsline{toc}{chapter}{Lista de Abrevia\c{c}\~oes}
\doublespacing  \begin{tabular}{l l}

DCC & Departamento de Ci\^encia da Computu\c{c}\~ao \\
UFJF & Universidade Federal de Juiz de Fora \\



\end{tabular}  \thispagestyle{empty}
%---------------------




%%%%%%%%%%%%%%%%%%%%%%%%%%%%%%%%%%%%%%%%%%%%%%%%%%%%%%%%%%%
%
%--------------In\'icio do Conte\'udo---------------------------
%
%
\pagestyle{ruledheader}
\chapter{Introdu\c{c}\~ao}
Este modelo pretende atender \`as necessidades de padroniza\c{c}\~ao dos trabalhos de monografia do Departamento de Ci\^encia da Computa\c{c}\~ao, Instituto de Ci\^encias Exatas da Universidade Federal de Juiz de Fora e servir de guia para alunos e professores.
 
Elaborado com base nas normas da ABNT, este modelo cont\'em a formata\c{c}\~ao essencial para apresenta\c{c}\~ao de trabalhos acad\^emicos, contempladas em  cinco partes:

O texto pode ser preparado usando LaTeX (ou TeX). Por favor, procure seguir as instru\c{c}\~oes para que as monografias do DCC possuam uma apar\^encia uniforme.

\section{Figuras}
A impress\~ao de monografias normalmente feita em tons de branco e preto. Portanto, evite fazer uso de fotografias coloridas, a menos que, quando transformadas em tons de cinza seus detalhes continuem vis\'iveis.

As figuras devem ser integradas no texto, centralizadas de acordo com as margens. Para testar a visibilidade dos detalhes de suas figuras, por favor, fa\c{c}a a gera\c{c}\~ao de um arquivo imagem de impress\~ao (postscript) e observem se todos os detalhes est\~ao perfeitamente vis\'iveis e os textos leg\'iveis.  As figuras devem ser numeradas e todas devem ter uma legenda explicativa. 

Tenha especial cuidado com figuras feitas diretamente com as ferramentas MSOffice. Se a figura ocupar uma p\'agina completa, certifique-se que esta n\~ao ultrapasse as margens. Evite colocar figuras e tabelas no formato paisagem, vide a figura ~\ref{fig:logoufjf}.

\begin{figure}[ht]
 \begin{center}
   \includegraphics{./figs/logoInstituicao.png} % logo-ufjf1.png: 196x111 pixel, 72dpi, 6.91x3.92 cm, bb=0 0 196 111
  \caption{Logotipo da Universidade Federal de Juiz de Fora}
  \label{fig:logoufjf}
 \end{center}
\end{figure}

\section{F\'ormulas e equa\c{c}\~oes}
Equa\c{c}\~oes e F\'ormulas devem ser colocadas em uma nova linha, centralizadas e numeradas consecutivamente para fins de refer\^encia, como pode ser observado na equa\c{c}\~ao ~\ref{eq:exemplo}.
\begin{equation}
\int_{0}^{\infty}f(x)dx = x - \frac{x^3}{3!} + \frac{x^5}{5!} + \cdots = sin(x)
\label{eq:exemplo}
\end{equation}

\section{Algoritmos} 
As listagens de c\'odigo de programas n\~ao s\~ao consideradas figuras, de modo que n\~ao necessitam ter legenda. Listagens de c\'odigo geralmente trechos de c\'odigo retirados de programa, como c\'odigos C, Java ou XML. Para fins de refer\^encia, as linhas do c\'odigo podem ser numeradas.

Por exemplo, o c\'odigo a seguir mostra uma classe Java, onde a linha 6 inicia um comando que se estende por diversas linhas. 


\lstset{tabsize=5,language=C,showstringspaces=false,basicstyle=\ttfamily\small,keywordstyle=\bf,breaklines=true}
\begin{singlespacing}
\begin{lstlisting}[frame=single,framexrightmargin=1pt,numbers=left]
import java.util.Random;
class Aleatorios {
  public static void main (String[] args) {
     Random qq=new Random();
     for (int k=1;k<10;k++)
        System.out.println(qq.nextInt(100) + "\n" + Math.random());
    }
}
\end{lstlisting}
\end{singlespacing}

 Algoritmos geralmente s\~ao apresentados como pseudo-c\'odigos, os quais possuem uma formata\c{c}\~ao formal conhecida dos livros de computa\c{c}\~ao. Diferentemente das listagens, os algoritmos costumam possuir legendas, como no algoritmo ~\ref{alg:exemplo} abaixo.

\begin{algorithm}
\label{alg:exemplo}
 \caption{Ler n\'umero e imprimir se \'e par ou n\~ao.}

     \Entrada{n\'umero, ($numero$).}
     \Saida{Se o n\'umero \'e par ou n\~ao}
     \Inicio{
         \textbf{ler} $numero$\;
         \eSe {$numero \% 2 = 0$} {
             \textbf{imprimir} $numero$, " par"\;
         } {
         \textbf{imprimir} $numero$, " impar"\;
         }
        }
\end{algorithm}


\section{Tabelas}

Tabelas necessitam de legenda superior, conforme mostrado na tabela~\ref{tab:modelos}.

\begin{table}[ht]
   \centering
   \caption{Porcentagem de modelos por marca}
   \label{tab:modelos}
   \begin{tabular}{| c | c |}
      \hline 
      Marca & Porcentagem \\
      \hline \hline 
      XPTO & 60\% \\
      \hline
      ZWY & 10\% \\
      \hline
      AWK & 10\% \\
      \hline
      HKL & 10\% \\
      \hline
      TPOI & 5\% \\
      \hline
      SSO & 5\% \\
      \hline
   \end{tabular} 
\end{table}

\section{Notas de rodap\'e}
As notas de rodap\'e s\~ao usadas ao longo do texto para esclarecimentos r\'apidos sobre algum conceito ou para refer\^encia a algum endere\c{c}o da web, como, por exemplo, na seguinte frase: "A W3C\footnote{http://www.w3.org} \'e o cons\'orcio regulador da Web".



\section{Cita\c{c}\~oes e refer\^encias}
Ao longo do texto, as cita\c{c}\~oes s\~ao feitas atrav\'es do formato (AUTOR, ANO). Ao final, a lista de refer\^encias deve ter o nome de Refer\^encias Bibliogr\'aficas, sem numera\c{c}\~ao de se\c{c}\~ao. n\~ao colocar quebra de p\'agina antes. 

Para inserir refer\^encias no documento Latex, se utilize do pacote \textsc{abntex2cite} e uso os comandos \textsc{cite} e \textsc{citeonline}. Com o comando \textsc{cite} as refer\^encias ficam como \cite{stojanovic2002} ou \cite{souza2010book, souza2008ismicka} em caso de refer\^encias consecutivas. Com o comando \textsc{citeonline} as refer\^encias ficam no formato de cita\c{c}\~ao usadas para seten\c{c}as que citam o autor: "Segundo \citeonline{zhang2007}...".


\section{Coment\'arios}
Durante o trabalho de confec\c{c}\~ao da monografia, o aluno ter\'a que enviar v\'arias vers\~oes para o orientador e este retornar\'e as vers\~oes com coment\'arios sobre o texto. Para facilitar esse trabalho de revis\~ao do aluno e orientador, pode-se incluir no texto coment\'arios do pacote \textsc{todonotes}. Os coment\'arios podem ser dispostos \textsc{na margem} \comment{Este \'e um coment\'ario na margem da p\'agina} ou ser do tipo \textsc{inline}.\comment[inline]{Este \'e um coment\'ario inline}

Caso voc\^e v\'a inserir uma figura no texto mas ainda n\~ao o fez, voc\^e pode informar isso ao seu orientador como no exemplo abaixo:
\missingfigure{A figura ainda ser\'a criada e inserida aqui.}

Com esse pacote \textsc{todonotes} voc\^e pode tamb\'em inserir em um local da sua monografia um \'indice de todos os coment\'arios que o seu orientador fez, para que voc\^e se lembre do que ainda tem que ser corrigido na monografia. Veja abaixo:

\begin{singlespacing}\listoftodos\end{singlespacing}


\comment[color=green!40]{Lembre-se: todos os coment\'arios e a lista de tarefas \textbf{devem} ser retirados do texto na vers\~ao que ser\'a enviada para a banca.}


\chapter{Texto de exemplo}
\section{A no\c{c}\~ao de fun\c{c}\~ao}
\indent\indent Iremos trabalhar com a no\c{c}\~ao intuitiva de fun\c{c}\~ao.
Uma defini\c{c}\~ao formal de fun\c{c}\~ao, na qual se faz uso da linguagem de
conjuntos e produtos cartesiano, ser\'a dada no Anexo I.

Uma fun\c{c}\~ao envolve um conjunto $A$, chamado de \textsc{dom\'inio},
um conjunto $B$ chamado de \textsc{contradom\'inio} e uma regra
denotada por $f:A \rightarrow B$, que nos diz como associar a cada
$a \in A$, um \'unico $f(a)=b \in B$, chamado de \textsc{valor de}
$f$ \textsc{no ponto} $a$ ou \textsc{imagem de} $a$ \textsc{pela
fun\c{c}\~ao} $f$.
\begin{example}\label{R2emR}
Seja $A=\R^2$, ou seja, os conjunto de todos os pares ordenados
$(x,y)$ tais que $x,y \in \R$ e seja $B=\R$ o conjunto dos n\'umeros
reais. Ent\~ao $f:A \rightarrow B$, definida para cada par $(x,y)$
por $f(x,y)=x$ \'e uma fun\c{c}\~ao, uma vez que, para cada par $(x,y)$,
corresponde um \'unico $x \in \R$.
\end{example}
\begin{example}
Se tomarmos $A=\R$ e $B=\R^2$ a regra $g:A \rightarrow B$,
definida para cada $x \in \R$ por $g(x)=(x,y)$ onde $y \in \R$,
n\~ao \'e uma fun\c{c}\~ao pois, para cada $x \in \R$, existem infinitos
pares ordenados $(x,y) \in \R^2$.
\end{example}

\begin{definition}
Duas fun\c{c}\~oes $f:A \rightarrow B$ e $g:C \rightarrow D$ s\~ao iguais
se as seguintes condi\c{c}\~oes s\~ao satisfeitas:
\begin{enumerate}
    \item $A=C$ e $B=D$;
    \item para cada $a \in A$, $f(a)=g(a)$.
\end{enumerate}
\end{definition}
\begin{definition}
Dada uma fun\c{c}\~ao $f:A \rightarrow B$, o subconjunto de $B$ formado
pelos elementos $b=f(a)$, com $a \in A$, \'e chamado de
\textsc{imagem} de $A$ por $f$, ou \textsc{imagem} de $f:A
\rightarrow B$.
\end{definition}
Usaremos $Im(f)$ ou $f(A)$ para denotar a imagem de $f:A
\rightarrow B$. Portanto, temos
$$
f(A)=\{b=f(a)\in B\;;\; a \in A\}.
$$

\section{Fun\c{c}\~oes Injetivas, Sobrejetivas e Bijetivas}

\begin{definition}
Seja $f:A \rightarrow B$ uma fun\c{c}\~ao. Dizemos que:
\begin{enumerate}
    \item $f$ \'e \textsc{injetiva} (ou \textsc{um-a-um}, ou \textsc{injetora}, ou uma
    \textsc{inje\c{c}\~ao}) sempre que $a \neq a'$ em $A$, $f(a)\neq
    f(a')$ em $B$;
    \item $f$ \'e \textsc{sobrejetiva} (ou \textsc{sobre}, ou \textsc{sobrejetora}, ou uma
    \textsc{sobreje\c{c}\~ao}) sempre que $f(A)=B$;
    \item $f$ \'e  \textsc{bijetiva} (ou \textsc{bijetora}, ou uma \textsc{bije\c{c}\~ao}) se $f$
    \'e injetiva e sobrejetiva.
\end{enumerate}
\end{definition}

Observe que, equivalentemente, podemos dizer que $f:A \rightarrow
B$ \'e injetiva sempre que $f(a)=f(a')$ implicar em $a=a'$.

\begin{example}
A fun\c{c}\~ao do exemplo (\ref{R2emR}) \'e sobrejetiva. De fato, dado $x'
\in \R$, para qualquer $y \in \R$, temos $f(x',y)=x'$. Observe
tamb\'em que tal fun\c{c}\~ao n\~ao \'e injetora pois, por exemplo
$f(2,1)=f(2,3)$, mas $(2,1)\neq (2,3)$.
\end{example}
\begin{definition}
Dados uma fun\c{c}\~ao $f:A \rightarrow B$ e $C\subseteq A$, definimos a
\textsc{restri\c{c}\~ao de} $f$ \textsc{a} $C$ como sendo a fun\c{c}\~ao $g:C
\rightarrow B$, definida por $g(c)=f(c)$, para todo $c \in C$.
Normalmente usa-se a nota\c{c}\~ao $f|_C$ para indicar a restri\c{c}\~ao de
$f$ a $C$.
\end{definition}
\begin{definition}
Sejam $f:A \rightarrow B$ uma fun\c{c}\~ao, $C \subseteq A$ e $D
\subseteq B$. Definimos:
\begin{enumerate}
    \item a \textsc{imagem (direta) de} $C$ por $f$ com sendo o subconjunto
    de $f(A)$ dado por
    $$
     f(C)=\{f(x)\;;\; x \in C \},
    $$
    \item  e a \textsc{imagem inversa de} $D$ por $f$ como sendo o subconjunto
    de $A$ dado por
    $$
    f^{-1}(D)=\{a \in A\;;\; f(a) \in D \}.
    $$
\end{enumerate}
\end{definition}

Note que a imagem inversa $f^{-1}(D)$, de um conjunto $D$ por uma
fun\c{c}\~ao $f$ pode ser o conjunto vazio, mesmo que $D$ n\~ao seja o
conjunto vazio. Por exemplo, considere a fun\c{c}\~ao $f:A \rightarrow
B$, onde $A=B=\R$, definida por $f(x)=|x|$. Ent\~ao, se $$D=\{x \in
\R\;;\; x<0\},$$ teremos que $f^{-1}(D)=\emptyset$, uma vez que
n\~ao existe $x \in \R$ tal que $|x|<0$.

\begin{proposition}\label{Img-direta}
Sejam $f:A \rightarrow B$ uma fun\c{c}\~ao e $C, D$ subconjuntos de $A$.
\begin{enumerate}
    \item[(a)] Se $C \subseteq D$, ent\~ao $f(C)\subseteq f(D)$;
    \item[(b)] $f(C \cup D)=f(C)\cup f(D)$;
    \item[(c)] $f(C \cap D) \subseteq f(C)\cap f(D)$.
\end{enumerate}
\end{proposition}
\begin{proof}\mbox{}
\begin{enumerate}
    \item[(a)] Se $y \in f(C)$, ent\~ao existe $c \in C$ tal que $y
    = f(c)$. Como $C \subseteq D$, segue que $c \in D$ e portanto
    $y=f(c) \in f(D)$, isto \'e, $f(C)\subseteq f(D)$.
    \item[(b)] Seja $y \in f(C \cup D)$. Ent\~ao existe $x \in C \cup
    D$ tal que $y=f(x)$. Logo $x \in C$ ou $x \in D$. Se $x \in
    C$, temos que $y=f(x) \in f(C)$. Mas se $x \in D$, teremos
    $y=f(x) \in f(D)$. Logo $y \in f(C) \cup f(D)$ o implica em
    $f(C \cup D)\subseteq f(C)\cup f(D)$.

    Por outro lado, como $C \subseteq C \cup D$, segue do item (a)
    que $f(C) \subseteq f(C \cup D)$. Da mesma maneira temos que $D
    \subseteq C \cup D$ o que implica que $f(D) \subseteq f(C \cup D)$.
    Portanto $f(C)\cup f(D)\subseteq f(C \cup D)$.
    \item[(c)] Como $C \cap D \subseteq C$, segue que $f(C\cap D)
    \subseteq f(C)$. Analogamente, temos $f(C\cap D)\subseteq
    f(D)$. Assim, podemos concluir que $f(C \cap D) \subseteq
    f(C)\cap f(D)$.
\end{enumerate}
\end{proof}

Devemos observar que em geral, no item \emph{(c)} da proposi\c{c}\~ao
(\ref{Img-direta}), n\~ao se pode substituir o sinal de inclus\~ao
pelo sinal de igualdade, ou seja nem sempre vale a inclus\~ao
oposta. Por exemplo, sejam $A=B=\R$ e $f:A \rightarrow B$ definida
por $f(x)=x^2$, $C=\{x \in \R\;;\; -1 \leq x <0\}$ e $D=\{x \in \R
\;;\; 0 <x \leq 1\}$. Ent\~ao
$$
f(C)=f(D)=\{x \in \R \;;\; 0 <x \leq 1\}=f(C)\cap f(D).
$$
Por outro lado $C \cap D= \emptyset$ o que implica em $f(C \cap
D)=\emptyset$. Portanto $$f(C)\cap f(D) \nsubseteq f(C \cap D).$$

\begin{proposition}\label{Img-inversa}
Sejam $f:A \rightarrow B$ uma fun\c{c}\~ao e $E, F$ subconjuntos de $B$.
\begin{enumerate}
    \item[(a)] Se $E \subseteq F$, ent\~ao $f^{-1}(E)\subseteq f^{-1}(F)$;
    \item[(b)] $f^{-1}(E \cup F)=f^{-1}(E)\cup f^{-1}(F)$;
    \item[(c)] $f^{-1}(E \cap F)=f^{-1}(E)\cap f^{-1}(F)$.
\end{enumerate}
\end{proposition}
\begin{proof}\mbox{}
\begin{enumerate}
    \item[(a)] Se $a \in f^{-1}(E)$, ent\~ao $f(a) \in E$.
    Como $E \subseteq F$, segue que $f(a) \in F$, isto \'e,
    $a \in f^{-1}(F)$. Portanto $f^{-1}(E)\subseteq f^{-1}(F)$.
    \item[(b)] Como $E \subseteq E \cup F$ e $F \subseteq E \cup
    F$, segue que $f^{-1}(E) \cup f^{-1}(F) \subseteq f^{-1}(E \cup
    F)$.

    Por outro lado, se $a \in f^{-1}(E \cup F)$, ent\~ao $f(a) \in E
    \cup F$, ou seja, $f(a) \in E$ ou $f(a) \in F$ e isto implica em
    $a \in f^{-1}(E)$ ou $a \in f^{-1}(F)$. Logo $a \in
    f^{-1}(E)\cup f^{-1}(F)$. Portanto $f^{-1}(E \cup F)
    \subseteq f^{-1}(E)\cup f^{-1}(F)$.
    \item[(c)] Como $E \cap F \subseteq E$ e $E \cap F \subseteq
    F$, temos que $f^{-1}(E \cap F)\subseteq f^{-1}(E)\cap
    f^{-1}(F)$.

    Reciprocamente, se $a \in f^{-1}(E) \cap f^{-1}(F)$, ent\~ao
    $f(a) \in E$ e $f(a) \in F$. Portanto, $f(a) \in (E \cap
    F)$, ou seja, $a \in f^{-1}(E \cap F)$. Logo $f^{-1}(E)
    \cap f^{-1}(F) \subseteq f^{-1}(E \cap F)$.
\end{enumerate}
\end{proof}

\section{Composi\c{c}\~ao de Fun\c{c}\~oes}
\begin{definition}
Sejam $f:A \rightarrow B$ e $g:B \rightarrow C$ duas fun\c{c}\~oes. A
composi\c{c}\~ao $g \circ f$ \'e a fun\c{c}\~ao de $A$ em $C$, definida por $(g
\circ f)(a)=g(f(a))$.
\end{definition}
\begin{example}\label{log-sen}
Considere as fun\c{c}\~oes $\log :\R_+ \rightarrow \R$ e $\mbox{sen}:\R
\rightarrow [0,1]$, onde $\R_+$ \'e o conjunto dos n\'umeros reais n\~ao
negativos. Ent\~ao $(\mbox{sen} \circ \log):\R_+ \rightarrow [0,1]$
\'e a fun\c{c}\~ao que a cada $x \in \R_+$ associa o valor
$\mbox{sen}(\log(x))$. Por outro lado, $(\log \circ \mbox{sen})$
nem sempre est\'a definida (n\~ao existe $\log( \mbox{sen}(x))$,
quando $x= \frac{3\pi}{2}$). Isso mostra que, em geral $f \circ g
\neq g \circ f$, isto \'e, a composi\c{c}\~ao de fun\c{c}\~oes \'e n\~ao-comutativa.
\end{example}

\begin{proposition}
Sejam $f:A\rightarrow B$, $g:B \rightarrow C$ e $h:C\leftarrow D$
fun\c{c}\~oes. Ent\~ao,
$$
h\circ(g \circ f)=(h \circ g)\circ f:A \rightarrow D.
$$
\end{proposition}
\begin{proof}
Basta provar que para cada $a \in A$, $h((g \circ f)(a))=(h \circ
g)(f(a))$. Mas,
$$
h((g \circ f)(a))=h(g(f(a)))=(h \circ g)(f(a)).
$$
\end{proof}
\begin{theorem}
Se $f:A \rightarrow B$ e $g:B \rightarrow C$ s\~ao ambas injetoras
(sobrejetora), ent\~ao $g \circ f$ tamb\'em \'e injetora (sobrejetora).
\end{theorem}
\begin{proof}
Vamos mostrar inicialmente que, se $f:A \rightarrow B$ e $g:B
\rightarrow C$ s\~ao ambas injetoras, ent\~ao $g \circ f$ tamb\'em o \'e.
Sejam $a_1, a_2 \in A$ tais que $(g \circ f)(a_1)=(g \circ
f)(a_2)$. Ent\~ao, $g(f(a_1))=g(f(a_2))$. Como $g$ \'e injetora, segue
que $f(a_1)=f(a_2)$. Mas como $f$ tamb\'em \'e injetora, temos
$a_1=a_2$, o que prova que $g \circ f$ \'e injetora.

Vamos mostrar agora que, se $f:A \rightarrow B$ e $g:B \rightarrow
C$ s\~ao ambas sobrejetoras, ent\~ao $g \circ f$ tamb\'em o \'e. Dado $c
\in C$, existe $b \in B$ tal que $g(b)=c$, uma vez que $g$ \'e
sobrejetora. Por outro lado, como $f$ tamb\'em \'e sobre, existe $a \in A$
tal que $f(a)=b$. Logo,
$$
(g \circ f)(a)=g(f(a))=g(b)=c.
$$
Portanto, $g \circ f$ \'e sobrejetora.
\end{proof}
\begin{corollary}\label{comp-bi}
Se $f:A \rightarrow B$ e $g:B \rightarrow C$ s\~ao ambas bijetoras,
ent\~ao $g \circ f$ tamb\'em \'e bijetora.
\end{corollary}

Dado um conjunto $A$, iremos denotar por $\id_A$ a fun\c{c}\~ao
identidade de $A$, ou seja, $\id_A:A \rightarrow A$ \'e a fun\c{c}\~ao
definida por $\id_A(a)=a$.

\begin{theorem}\label{inversa-func}
Se $f:A \rightarrow B$ \'e uma fun\c{c}\~ao bijetora, ent\~ao existe uma
\'unica fun\c{c}\~ao tamb\'em bijetora, $g:B \rightarrow A$, tal que $g
\circ f = \id_A$ e $f \circ g = \id_B$. A fun\c{c}\~ao $g$ \'e chamada de
inversa de $f$ e \'e geralmente, denotada por $f^{-1}$.
\end{theorem}
\begin{proof}
Dado $b \in B$, como $f$ \'e bijetora existe $a \in A$ tal que
$f(a)=b$ e se $a' \in A$ tal que $f(a')=b$, ent\~ao $a'=a$, ou seja,
dado $b \in B$, existe um \'unico $a \in A$, satisfazendo $f(a)=b$.
Denotando tal $a$ por $g(b)$, definimos uma fun\c{c}\~ao $g:B
\rightarrow A$. Vamos mostrar que tal fun\c{c}\~ao satisfaz as
propriedades do teorema.

Temos que $(f \circ g)(b)=f(g(b))=f(a)=b$, ou seja $f \circ
g=\id_B$. Por outro lado $(g \circ f)(a)=g(f(a))=g(b)=a$, ou seja,
$g \circ f=\id_A$.

Vamos provar agora que $g$ tamb\'em \'e bijetora. Segue da defini\c{c}\~ao
de $g$ que a mesma \'e sobrejetora e portanto, basta mostrarmos que
a mesma e injetora. Sejam $b_1,b_2 \in B$ tais que
$g(b_1)=g(b_2)$. Ent\~ao, como $f \circ g=\id_B$, temos
$$
b_1=\id_B(b_1)=(f \circ g)(b_1)=f(g(b_1))=f(g(b_2))=(f \circ
g)(b_2)=\id_B(b_2)=b_2.
$$
o que mostra que $g$ \'e injetora.

Suponha agora que existe $h:B \rightarrow A$ satisfazendo $h \circ
f = \id_A$ e $f \circ h = \id_B$. Ent\~ao,
$$
h=h \circ \id_B= h \circ (f \circ g)= (h \circ f) \circ g)=\id_A
\circ g= g.
$$
Logo $g$ \'e \'unica.
\end{proof}
\begin{remark}
Note que se $g=f^{-1}$, ent\~ao $f=g^{-1}$.
\end{remark}

Se uma fun\c{c}\~ao $f:A \rightarrow B$ admite uma inversa, dizemos que
a mesma \'e \textsc{invers\'ivel}. Portanto, pelo teorema
(\ref{inversa-func}), toda fun\c{c}\~ao bijetora \'e invers\'evel. O pr\'oximo
resultado mostra que a rec\'iproca desta afirma\c{c}\~ao tamb\'em \'e
verdadeira.

\begin{proposition}\label{comp-bije}
Se $f:A \rightarrow B$ e $g:B \rightarrow A$ s\~ao fun\c{c}\~oes que
satisfazem $g \circ f = \id_A$ e $f \circ g = \id_B$, ent\~ao ambas
s\~ao bijetoras.
\end{proposition}
\begin{proof}
Vamos provar que $g$ \'e bijetora (a prova para $f$ \'e an\'aloga).
Sejam $b_1, b_2 \in B$ tais que $g(b_1)=g(b_2)$. Ent\~ao
$f(g(b_1))=f(g(b_2))$. Como $f \circ g = \id_B$, temos que
$\id_B(b_1)=\id_B(b_2)$, ou seja, $b_1=b_2$, provando que $g$ \'e
injetora.

Dado $a \in A$, temos
$$
a=\id_A(a)=(g \circ f)(a)=g(f(a)).
$$
Assim, tomando $b=f(a)$ teremos $g(b)=a$ e isto prova que $g$ \'e
sobrejetora.
\end{proof}



%%%%%%%%%%%%%%%%%%%%%%%%%%%%%%%%%%%%%%%
\singlespacing
\bibliographystyle{abntex2-alf}
\bibliography{referencias}

%%%%%%%%%%%%%%%%%%%%%%%%%%%%%%%%%%%%%%%%%%%%%%%%

\end{document}
